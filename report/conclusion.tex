% !TeX root = main.tex

\section{总结与展望}

基于来自 Landsat 和 Sentinel-2 的卫星遥感影像数据, 本文选取 NDVI 为指标, 在时间和空间两个维度上对清华校园内 1984 年 -- 2022 年间的绿化情况进行持续性的分析.

从 VI 的时间序列出发, 在一年的尺度范围内, 绿化的季节性变化与一般植被在生物学上的季节变化相一致.
在更长的时间范围内, VI 的变化也能够作为衡量校园绿化整体状况的指标.
从 VI 的空间分布出发, 结合历史上的校园规划图可以发现 VI 的空间分布与校园规划中的绿化分布基本吻合, 这进一步说明了选取 VI 为衡量绿化指标的合理性.

来自卫星的遥感数据具有很可观的数量和质量, 但对于校园绿化复杂的植被分布仍然是不足的.
例如, 不同植被的区分、垂直空间上的分布、生长状况等细节难以使用单一的遥感数据监测.
为此, 可以适当引入人力调查、小型监测站、无人机观测等地面观测技术, 进一步提升观测数据的细度和扩展可观测数据的维度.
结合生物学领域的知识和技术, 能够对校园绿化进行更完善和全面的监测.
